\chapter{Glossary and Acronyms}
\label{app_glossary}

Care has been taken in this thesis to minimize the use of jargon, but this cannot always be achieved.  This appendix defines certain terms used in a glossary, and contains a table of acronyms and initialisms used along with their meanings.

% Glossary {{{
\section{Glossary}
\label{sec_glossary}

\begin{itemize}

% \item
%     \textbf{Cosmic-Ray Muon} 
%         (\textbf{CR $\mu$}) -- A muon coming from the abundant energetic particles originating outside of the Earth's atmosphere.

\glossaryvar{Beam Energy}{$\Ebeam$}{The energy available to each $\lel$ or $\alel$ in the initial collision ($\Ebeam = \half \Ecm$).}
\glossaryitem{Breit-Wigner}{A distribution commonly used to model resonance production.}
\glossaryitem{Cabbibo Suppression}{Decays which proceed through a disfavored quark decay channel (e.g., $\qcharm \rightarrow \qdown$ instead of $\qcharm \rightarrow \qstrange$).}
\glossaryvar{Center-of-Mass Energy}{$\Ecm$}{The total energy available from a $\ee$ collision.}
\glossaryitem{Cross Section}{The production rate for a specific group of particles as a function of center-of-mass energy.}
\glossaryitem{Decay Mode}{A specific group of particles produced from the decay of a parent particle (e.g., $\DOmodeA$).}
\glossaryitem{Feynman Diagram}{A visual representation of a particle decay used to simplify the mathematical description and calculations.}
\glossaryitem{Form Factor}{A function which reflects decay properties of a particle, but does not necessarily capture all the underlying physics.}
\glossaryitem{Interference}{The overlapping of wave amplitudes in the particle fields which modifies the overall shape.}
\glossaryitem{Lifetime}{The average amount of time before a specific type of particle decays.}
\glossaryvar{Luminosity}{$\lum$}{The rate of collisions produced by the accelerator.}
\glossaryitem{Multiplicity}{The number of tracks occurring in a specific decay mode or the total event.}
\glossaryitem{Resonance}{An unstable, bound-state particle with a generally short mean lifetime (${\sim}10^{23}$).}
\glossaryitem{$\SUthree, \; \SUtwo, \; \Uone$}{Group theory representations which are used to describe the interactions of the fundamental forces.}
\glossaryvar{Virtual Photon}{$\gamma^*$}{A photon modeled in the intermediate particle exchange of a Feynman diagram which does not have a well-defined mass.}

\end{itemize}

\pagebreak

% Acronyms 
\section{Acronyms / Initialisms}
\label{sec:acronym}

% Table formatting

% Heading for the first page
\begin{longtable}{p{0.25\textwidth} p{0.75\textwidth}}
\label{tab:acronyms} \\

\toprule
Name & Meaning \\
\midrule
\endfirsthead

% Heading for all subsequent pages
\multicolumn{2}{l}{\textit{\tablename\ \thetable{} -- Continued from previous page}} \\
\toprule
Acronym & Meaning \\
\midrule
\endhead

% Footer for each page that wraps over to the next
\multicolumn{2}{r}{\textit{Continued on next page}} \\
\bottomrule
\endfoot

% Footer for the end of the table
\bottomrule
\endlastfoot

% End table formatting

%CR$\mu$ & Cosmic-Ray Muon \\
    ADC    & Analog-to-Digital Conversion \\
    BEPCII & The second Beijing Electron-Position Collider \\
    BESIII & The third Beijing Spectrometer \\
    BOSS   & BESIII Offline Software System \\
    EMC    & Electromagnetic Calorimeter \\
    FSR    & Final State Radiation \\
    IHEP   & Institute of High Energy Physics \\
    ISR    & Initial State Radiation \\
    MC     & Monte Carlo \\
    MDC    & Multi-Layer Drift Chamber \\
    MUC    & Muon Identifier \\
    PMTs   & Photomultipler Tubes \\
    RPC    & Resistive Plate Counter \\
    TDC    & Time-to-Digital Conversion \\
    ToF    & Time-of-Flight System \\
\end{longtable}
