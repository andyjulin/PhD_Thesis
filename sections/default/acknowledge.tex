Wow, okay, where to start...

One of the most fortunate aspects of my graduate career has been having incredible professors teach my courses.
More than just hoping I learned the material, every single one of them seemed to care about helping prepare me as a successful researcher.
I can't count how many times I had questions during lecture or office hours, and yet, never once did I get the feeling like I was bothering or frustrating the professor (and I'm sure plenty of the questions could have!).
This also speaks more to the department as a whole, as I've always felt included by the faculty here.
I can vividly recall a time just prior to making a decision about graduate school, and having multiple professors greet me in the halls, and inquire about how I was doing personally.
This aspect was one of the main factors in continuing my education here at the University of Minnesota, and I have been pleased to see it only become more prevalent as a graduate student.

But the professors are only half of the department; the graduate students also have a substantial impact on your daily life.
Again, I have been unbelievably fortunate to have also met a tremendous number of friends through this program.
There are plenty of horror stories around the country about competitiveness and exclusivity between graduate students, but this has never been representative of this department.
Instead, I developed many friends through the `joys' of working through homework or while preparing for things like the Graduate Written Exam.
I'm sure I will miss names if I try, but if we ever worked on homework together in Tate 216, know that you were vital to helping me through those first few years!

More than just the incoming class of 2011, I was also immediately welcomed by the more senior graduate students.
Getting to learn from their experiences helped immensely for navigating the treacherous first years of graduate school.
Even with their expertise, I never felt a sense of superiority coming from these older students.
Instead, we were both equals on different parts of the path, with new ideas and experiences to share in order to help each of us flourish.
This is an aspect I also tried to pass on with the newer graduate students, each of which were interested to hear about my experiences, and were eager share their perspectives with me.
Again, I will miss names if I try, but if we ever went to Chipotle together on Fridays, know that you were part of the most interesting conversations of my life! 

I have also been extremely lucky to have found such a perfect fit with my research group, BESIII.
While still no one really knows what we do (and I'll admit, I had no idea when I first joined either), this experience has been absolutely essential in preparing me as a data analyst.
Much of this is due to my advisor, Professor Ron Poling, who has been second to none in helping me becoming a successful researcher.
No one in the department possesses a greater combination of experience and insight crucial to helping students develop the necessary skills for a successful career.
He uses every opportunity as a learning experience, and best of all, works to transform a myriad of techniques into an overall perception of how to successfully analyze data.
From the considerable guidance he has provided me with over these past six years, I feel confident about being prepared for my future.

There has also been a number of other very impactful people in BESIII throughout my time here.
All of the previous members, such as Nick Howell, Nick Smith, and Derrick Toth, were all very useful for helping me understand a variety of new techniques, especially ones related to computing.
Professor Dan Cronin-Hennessy was also instrumental for understanding the details of the experiment and running analysis code, not to mention forming the basis of the analysis contained in this thesis.
I also can recall a myriad of times where I asked Jianming Bian for assistance, whether about analysis techniques, or detector components, or physics processes.
No matter what the subject, it seemed like he was always the first person to consult, and virtually always provided useful insight.

It has also been an absolute privilege to work with the current members in the group.
Hajime Muramatsu has also answered questions about virtually every aspect of my analysis, and suggested a variety of important other aspects to consider.
I also cannot thank him enough for guiding me through our travels together during my first visit to Beijing!
People like Dan Ambrose and Alex Gilman have also been not just enjoyable office mates, but also great friends.
It has been wonderful to be able to discuss anything about physics, as well as explicitly non-science things when we all need a break from research.
I look forward to seeing what path each of us takes, and the many ways they will contribute to the field of physics.

The continued encouragement from my family and friends has also been vital to making it through these last six years.
I have to generously thank both of my parents for helping proofread this thesis even though they said, ``It was like reading something written in a different language.''
They have always shown a willingness to assist and help support me in any way possible, and I fully appreciate all that they have done for me.
I also have to thank my brother and sister, Thomas and Angela, for their increasingly inventive descriptions of what they think I do!
Although, these usually pale in comparison to those suggested by my friends, who generally assume I now have mastery over all of time and space...

Lastly, I have to give the utmost acknowledgement to my amazing wife, Caitlin.
It would take countless pages to describe all the ways she has been supportive of me during our time together.
This is also through the struggles of working towards her own PhD in Physics, as well as us being 2000 miles away from each other for the past five years.
And even with all of this going on, no one is more readily available to help, nor does anyone feel closer to me than her.
I look forward to our new lives together, and am eager to see all of the ways she helps benefit the world throughout our future together.
My only hope is I can strive to be anywhere near the amazing person she continues to be!

Everyone else I've missed, know that you are awesome, and I have been absolutely thrilled to know you during my time here.
It's impossible to properly credit all of the people who have shaped my graduate career, but hopefully you all know how much of a difference you've made in my life!
I cannot thank you enough!

