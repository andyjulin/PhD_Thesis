\chapter{Conclusion}
\label{ch_conclusion}

Using the high statistics $\ee \rightarrow \psipp$ collision data available at BESIII, we have measured $\xsecpsipptoDDbar$ as a function of center-of-mass energy more precisely than ever before.
In this analysis, we have verified the necessity of including the effect of interference from the $\psip$.
Namely, the vanishing Born cross section near \SI{3.81}{\GeV} can only be fitted with destructive interfere.
Our measured $\psipp$ parameters differ from world averages while being consistent with the much less precise measured values from KEDR.


While the results for the $\psipptoDDbar$ cross section are significant, there are certain aspects which indicate the need for future study.
Most notably, while the form factors used both show excellent agreement in the peak region of the $\psipp$, several points at higher energies, namely in the range \SIrange{3.81}{3.82}{\GeV}, show significant discrepancy.
However, the difficulty fitting the high-side energy dependence has minimal effect on measuring the parameters of the $\psipp$, which are dominated by the cross section shape near the peak.
A more sophisticated analysis would improve the model in the higher energy region, such as by using a Breit-Wigner shape for the $\psi(4040)$ instead of a constant parameter, or by exploring potential contributions from other resonances, such as near \SI{3.9}{\GeV}.
Other improvements would require minimizing the uncertainty on the cross section parameters involved in the fit, namely the meson radii.
The question of how to incorporate Coulomb interactions, if at all, also remains open.


The information provided by the $\DDbar$ cross section has also allowed us to investigate $\BFpsipptononDDbar$.
As it is not feasible to directly distinguish between $\qqbar$ and $\DDbar$ events produced above the open-charm threshold, obtaining this component must use information from below this region.
Based on the data available at BESIII, our methodology for this process involved extrapolating from center-of-mass energies just below the $\psip$ resonance.
However, this procedure is heavily reliant on our understanding of the $\psip$ cross section shape.
Without additional experimental information, the final results are not well constrained.
Instead, the best we can provide at this time are bounds on the $\nonDDbar$ branching fraction.
The results are consistent with previous measurements from BESII, though the uncertainties on their measurements are large.
However, the $\nonDDbar$ results presented here are preliminary, and should not be quoted as official BESIII results.


Better understanding the $\psipptononDDbar$ branching fraction requires a precise study of the $\psip$ cross section in the range of the continuum data.
BESIII has plans for data taking in this region during the 2017-2018 run.
This will help determine the effects of interference between the $\psip$ resonance and the continuum region around it.
Once precise $\psip$ cross section values are obtained at each of the continuum energy points, our analysis will be updated to produce a measurement of the branching fraction based on the current BESIII $\psipp$ data sample.
It remains to be seen, however, if the understanding of the continuum subtraction will be sufficient to reduce the systematic uncertainty appreciably from the current preliminary result.
% However, this relies on the assumption the $\qqbar$ component scales smoothly with energy over this range.
% If this is impacted by the presence of either the $\psip$ or $\psipp$ resonances over the extrapolation, an alternative strategy for determining the $\qqbar$ component will need to be employed.


For now, we have found very precise measurements of $\xsecpsipptoDDbar$ and $\sigma(\ee \rightarrow (\text{hadrons}))$ over the energy range near the $\psipp$.
The former determines multiple parameters of the $\psipp$, such as $\Mpsipp, \, \Gpsipp,$ and $\GeepsipptoDDbar$.
Each of these can immediately benefit other analyses throughout this region.
While not as imminent, combining our results with additional knowledge of the $\psip$ cross section should also lead to a more precise determination of $\BFpsipptononDDbar$.
This not only can be used to determine the value of $\Geepsipp$, but will also greatly benefit theories about strong interactions involving mixed state resonances.


