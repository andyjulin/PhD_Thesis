\chapter{Theoretical Background}
\label{ch_theory}

\section{Standard Model}
\label{sec_standard_model}

Developed throughout the 1960s and 1970s, the Standard Model provides the most complete description of observable matter in the universe to date.
It is a classification of all confirmed subatomic particles currently known, and predicts the most accurate results of any scientific theory ever measured.
Each of the Electromagnetic, Weak, and Strong fundamental forces are well described by this formulation.


This is not to say the theory is complete, however.  
For example, the discovery of neutrino oscillations disproves the Standard Model assumption that they are massless.
Additionally, the remaining fundamental force, Gravity, is completely absent from this theory. 
This omission, along with the aspects such as dark matter or dark energy, remain major hinderances in constructing a unified theory.
However, since gravitational effects are negligible on the scale of the masses of fundamental particles, it will be ignored in the discussions that follow.


There are two primary groups contained in the Standard Model, bosons and fermions. 
This division is based off the Spin Statistics theorem, where bosons have integer spins, and fermions have half-integer values.
Because of these values, the Pauli Exclusion principle restricts fermions from occupying the same spatial state, and thus there are restrictions on their spatial density.
Bosons, however, do not have this restriction, and can have any number occupying the same space.
Thus, fermions are typically more tangible matter (such as electrons), while bosons typically represent the forces interacting between them (such as photons).


\subsection{Fermions}
\label{ssec_fermions}

The fermions are divided by their interaction types into two major groups, quarks ($\quark$) and leptons ($\lepton$).
Each of these groups contains six particles and their six corresponding antiparticle (12 entities total).
These can also be categorized into three generations, which aligns particles with the same electric charges, but greatly differing masses.  
For example, the up ($\up$), charm ($\charm$), and top ($\top$) quarks all have an electric charge of +2/3, but $\top$ is many orders of magnitude more massive than $\up$.
In \ref{img_standard_model}, the rows indicate particles with the same electric charge, while the columns represent each generation of particles.


While all fermions interact both electromagnetically and weakly, only the quarks interact strongly.
With the strong interaction potential being very high at large separations, quarks cannot exist as isolated particles.
This behavior is known as confinement, and is why quarks are only found in nature as groups of particles called hadrons.
The most common types of hadrons exist as quark-antiquark pairs, known as mesons, or as groups of three quarks (or antiquarks), known as baryons.
There are, however, indications of more exotic combinations of quarks, such as tetra- ($\quark\quark\aquark\aquark$) or penta-quark ($\quark\quark\quark\quark\aquark$) states seen by recent experiments.


Strong interactions are associated with a corresponding conserved quantity known as color charge. 
Each quark carries one of three values for color charge, red ($\cRed$), green ($\cGreen$), or blue ($\cBlue$).
The possible colors for antiquarks are opposite of these values ($\acRed, \acGreen,$ and $\acBlue$).
In order for quarks to form observable hadrons, the total values of their constituents must be colorless.
The colorless combinations are $\cRed \acRed, \cGreen \acGreen, $ or $\cBlue \acBlue$ for mesons and $\cRed \cGreen \cBlue$ or $\acRed \acGreen \acBlue$ for baryons.
As color charge is not observable in nature, the number of possible hadron combinations are effectively tripled, due to each quark having three possible values.


\subsection{Bosons}
\label{ssec_bosons}

For each of the three forces included in the Standard Model, there are accompanying gauge bosons.  
These include the photon ($\photon$) for Electromagnetic force, the $\W$ and $\Z$ for the Weak force, and the gluon ($\gluon$) for the Strong force.
There is also the Higgs boson ($\Higgs$), which unifies the Electromagnetic and Weak forces, and whose interactions with other particles is responsible for their mass.


Both the photon and gluon are massless, while the $\W$ and $\Z$ have masses around 70 and 80 $\GeV$ respectively.
This means the interaction ranges of the Electromagnetic and Strong forces are effectively infinite, while the Weak force is extremely short ranged.
However, due to confinement, the strong interaction is generally limited to short distances in nature.
This range is on the order of the proton radius, around $10^{-15}$ \m. 


Each of the gauge bosons are a spin-1 vector boson, which means there are three available polarization states (-1, 0, +1).  
However, since the photon and gluon are both massive, gauge invariance requires these to have transverse polarizations.
This means the 0 state is eliminated, and there are only two polarization states for each.
The Higgs boson is the only known fundamental spin-0 particle, which means it has only one polarization state.

 
Though they are both massless, the photon and gluon have another important distinction in their interactions.
The photon, the mediator of the Electromagnetic force, has no electrical charge.
In contrast, the gluon, the mediator of the Strong force, also carries a color charge.
There are eight possible color combinations which a gluon may possess, which are typically expressed using the Gell-Mann representation of $\SUthree$.
With this basis, each gluon is linearly independent, and no combination of gluons can be used to form a color singlet state.
This inclusion of color by the force carrier makes QCD significantly more complex than QED.
In fact, carrying color charge means gluons can also interact with each other directly, leading to certain theoretical states such as glueballs. 


\section{Charmonium}

\section{OZI Suppression}

\section{Cross Sections}

\section{Breit-Wigner Formulation}