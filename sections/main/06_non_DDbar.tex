\chapter{Measurement of Hadronic Production and $\Gamma(\psipp \rightarrow \text{non-}\DDbar)$}
\label{ch:non_DDbar}


\section{Data and Monte Carlo Samples}
\label{sec:non_DDbar_data_samples}

\subsection{Data Samples}
\label{ssec:data_samples_non_DDbar}

The data used for this analysis was also produced by BEPCII and collected by BESIII.
The samples used include continuum data taken at \SI{3.650}{\GeV} in 2009, as well as multiple other continuum points taken around this energy in 2013.
We also use Round 1 (R1) and Round 2 (R2) of the high-statistics $\psipp$ data taken in 2010 and 2011, respectively.
Each of these samples, and their measured luminosities, can be seen in \Cref{tab:data_samples_non_DDbar}.
The values of luminosity were measured during a previous version of this analysis using the procedure described in \Cref{ssec:luminosity_measurement}.
The labels given to each continuum point are the intended energy targets, but may differ from their true values.
In addition to these datasets, the scan data described previously (\Cref{ssec:data_samples}) is also used.


\subsection{Center-of-Mass Energy Measurement}
\label{ssec:energy_measurement_non_DDbar}

As before, a precise measurement of each energy point is vital to the accuracy of the final results.
Most notably, due to the rapidly increasing shape of the $\psip$ cross section near the high end of the continuum points, the value of the 3671 (New) point has a dramatic effect on the $\qqbar$ extrapolation.
Following the procedure of \Cref{ssec:energy_measurement}, we measured the $\Ecm$ value of each continuum point.
This resulted in a \SIrange{4}{6}{\MeV} shift downwards for each of the new continuum points, but virtually no shift for the old continuum point.
The energy measurements are shown in \Cref{tab:data_samples_non_DDbar}.


\begin{table}[H]
\centering
\renewcommand\arraystretch{1.0}
\begin{tabular}{l|c r}
\hline
Sample Name & $\Ecm$ [\si{\GeV}] & Luminosity \si{\invpb} \\
\hline
3500 (New) & 3.496 & \num{  3.680 \pm 0.009} \\
3542 (New) & 3.538 & \num{  3.481 \pm 0.009} \\
3600 (New) & 3.596 & \num{  0.395 \pm 0.019} \\
3650 (New) & 3.644 & \num{  5.420 \pm 0.009} \\
3671 (New) & 3.665 & \num{  4.669 \pm 0.009} \\
3650 (Old) & 3.650 & \num{ 44.334 \pm 0.009} \\
3773 (R1)  & 3.773 & \num{926.922 \pm 0.092} \\
3773 (R2)  & 3.773 & \num{1978.92 \pm 0.091} \\
\hline
\end{tabular}
\caption{Data samples used for the inclusive measurement.}
{While the 3600 (New) sample was intended to be similar in luminosity to the other continuum points, accelerator issues inhibited the data collection procedure. 
There is also a sizable difference in total luminosity between the new continuum points and the other datasets used in this analysis.}
\label{tab:data_samples_non_DDbar}
\end{table}


\section{Event Selection}
\label{sec:non_DDbar_event_selection}

In order to select the number of hadronic events in each sample, we apply a variety of cuts.
For charged tracks in the MDC, these include the cuts shown in \Cref{tab:charged_cuts_non_DDbar}.

\begin{table}[H]
\centering
\renewcommand\arraystretch{1.0}
\begin{tabular}{c| r@{$\; < \;$}l}
\hline
Vertex ($xy$) & $V_{xy}$ & \pp \SI{1}{\cm} \\
Vertex ($z$)  & $|Vz|$   & \SI{10}{\cm} \\
MDC Angle & $|\cos\theta|$ & 0.93 \\
\hline
\end{tabular}
\caption{Selection cuts on charged tracks used to count hadronic events.}
\label{tab:charged_cuts_non_DDbar}
\end{table}

For neutral tracks in the EMC, these include the cuts shown in \Cref{tab:neutral_cuts_non_DDbar}.

\begin{table}[H]
\centering
\renewcommand\arraystretch{1.0}
\begin{tabular}{c|l r}
\hline
Minimum Energy (Barrel) & $E_{\text{EMC}} > \SI{25}{\MeV}$ & $(|\cos\theta| < 0.80)$ \\
Minimum Energy (Endcap) & $E_{\text{EMC}} > \SI{50}{\MeV}$ & $(0.86 < |\cos\theta| < 0.92)$ \\
TDC Timing & $(0 \leq t \leq 14) \times \SI{50}{\ns}$ & \\
\hline
\end{tabular}
\caption{Selection cuts on neutral tracks used to count hadronic events.}
\label{tab:neutral_cuts_non_DDbar}
\end{table}


To reject background events of $\bhabha$ or $\ee \rightarrow \gamma\gamma$, we also employ cuts related to the most energetic and highest momentum tracks in the event.
These are listed in \Cref{tab:bhabha_cuts_non_DDbar}.

\begin{table}[H]
\centering
\renewcommand\arraystretch{1.0}
\begin{tabular}{l|l@{}l l@{}l}
\hline
\multirow{4}{*}{Highest Energy}   & $\cosmax_+ <  $ & 0.8                            & \multirow{2}{*}{($\Ntrk$} & \multirow{2}{*}{ = 2)} \\
                                  & $\cosmax_- > -$ & 0.8                            & & \\
\cline{2-5}
                                  & $\cosmax_+ <  $ & 0.8 or $\pEmax_+ \leq 0.3$     & \multirow{2}{*}{($\Ntrk$} & \multirow{2}{*}{ = 3, 4)} \\
                                  & $\cosmax_- > -$ & 0.8 or $\pEmax_- \leq 0.3$     & & \\
\hline
\multirow{2}{*}{Highest Momentum} & \multicolumn{2}{l}{$0.8 \leq \Epmax_+ \leq 1.1$} & & \\
                                  & \multicolumn{2}{l}{$0.8 \leq \Epmax_- \leq 1.1$} & & \\
\hline
\end{tabular}
\caption{Selection cuts to remove Bhabha and two-photon backgrounds.}
{The $_+$ and $_-$ denote positively and negatively charged tracks, respectively.  The $^{\text{max}}$ notation indicates the highest energy or momenta track for the corresponding charge.  The energy cuts depend on the total number of charged tracks in the event, $\Ntrk$.}
\label{tab:bhabha_cuts_non_DDbar}
\end{table}

After applying these sets of cuts, there are three groups of cuts which are considered: Standard (SHAD), Loose (LHAD), and Tight (THAD).
For the nominal procedure, SHAD is used, while LHAD and THAD are for systematic consideration.
The cuts included in each of these sets are shown in \Cref{tab:shad_cuts_non_DDbar,tab:lhad_cuts_non_DDbar,tab:thad_cuts_non_DDbar}.
These apply to the number of charged tracks ($\Ntrk$), the visible energy ($\Evis$), the total visible momentum in the $z$-direction ($\pzvistot$), the maximum shower energy ($\Eemcmax$), and the total shower energy ($\Eemctot$).
Here, `visible' refers to the sum over charged and neutral tracks.

\begin{table}[H]
\centering
\renewcommand\arraystretch{1.0}
\begin{tabular}{l|r@{ }l l}
\hline
Number of Tracks                     & $\Ntrk$ & $ > 2$               &                  \\
\hline
Visible Energy                       & $\EvisE$ & $ > 0.3$            &                  \\
\hline
Visible Momentum                     & $\pzEvis$ & $ < 0.6$           & $(\Ntrk = 3, 4)$ \\
\hline
Maximum Shower Energy                & $\EemcEmax$ & $ < 0.75$           & $(\Ntrk = 3, 4)$ \\
\hline
\multirow{2}{*}{Total Shower Energy} & \multicolumn{2}{c}{$0.25 < \EemcEtot < 0.75$} & $(\Ntrk = 3)$ \\
                                     & \multicolumn{2}{c}{$0.15 < \EemcEtot < 0.75$} & $(\Ntrk = 4)$ \\
\hline
\end{tabular}
\caption{Standard selection cuts (SHAD) for counting hadronic events.}
\label{tab:shad_cuts_non_DDbar}
\end{table}

\begin{table}[H]
\centering
\renewcommand\arraystretch{1.0}
\begin{tabular}{l|r@{ }l l}
\hline
Number of Tracks                       & $\Ntrk$ & $ > 1$               &                  \\
\hline
\multirow{2}{*}{Visible Energy}        & $\EvisE$ & $ > 0.4$            & $(\Ntrk = 2)$    \\
                                       & $\EvisE$ & $ > 0.3$            & $(\Ntrk \geq 3)$ \\
\hline                                 
\multirow{2}{*}{Visible Momentum}      & $\pzEvis$ & $ < 0.3$           & $(\Ntrk = 2)$    \\
                                       & $\pzEvis$ & $ < 0.6$           & $(\Ntrk = 3, 4)$ \\
\hline
\multirow{2}{*}{Maximum Shower Energy} & $\EemcEmax$ & $ < 0.50$           & $(\Ntrk = 2)$    \\
                                       & $\EemcEmax$ & $ < 0.75$           & $(\Ntrk = 3, 4)$ \\
\hline                                 
\multirow{2}{*}{Total Shower Energy}   & \multicolumn{2}{c}{$0.25 < \EemcEtot < 0.75$} & $(\Ntrk = 2, 3)$ \\
                                       & \multicolumn{2}{c}{$0.15 < \EemcEtot < 0.75$} & $(\Ntrk = 4)$    \\
\hline
\end{tabular}
\caption{Loose selection cuts (LHAD) for counting hadronic events.}
\label{tab:lhad_cuts_non_DDbar}
\end{table}

\begin{table}[H]
\centering
\renewcommand\arraystretch{1.0}
\begin{tabular}{l|r@{ }l l}
\hline
Number of Tracks                     & $\Ntrk$ & $ > 3$               &                  \\
\hline
Visible Energy                       & $\EvisE$ & $ > 0.4$            &                  \\
\hline
Visible Momentum                     & $\pzEvis$ & $ < 0.6$           & $(\Ntrk = 4)$ \\
\hline
Maximum Shower Energy                & $\EemcEmax$ & $ < 0.75$           & $(\Ntrk = 4, 5)$ \\
\hline
\multirow{2}{*}{Total Shower Energy} & \multicolumn{2}{c}{$0.15 < \EemcEtot < 0.75$} & $(\Ntrk = 4)$ \\
                                     & \multicolumn{2}{c}{$0.00 < \EemcEtot < 0.75$} & $(\Ntrk = 5)$ \\
\hline
\end{tabular}
\caption{Tight selection cuts (THAD) for counting hadronic events.}
\label{tab:thad_cuts_non_DDbar}
\end{table}

\section{Background Subtraction}
\label{sec:background_subtraction}

\section{Efficiency Extrapolation}
\label{sec:efficiency_extrapolation}

\section{$\DDbar$ Correction}
\label{sec:DDbar_correction}

\section{Hadron Counting}
\label{sec:hadron_counting}

\section{Results}
\label{sec:non_DDbar_results}
