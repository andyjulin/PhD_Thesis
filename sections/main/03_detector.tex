\chapter{Detector and Related Systems}
\label{ch:detector}

All data used for this analysis were collected at the third Beijing Spectrometer (BESIII), located in Beijing, China, at the Institute of High Energy Physics (IHEP) campus.
This detector records $\ee$ collision events provided by the second Beijing Electron-Positron Collider (BEPCII).
The target energies for these collisions focus on $\ltau$ and $\qcharm$ production in the range of about \SIrange{2.0}{4.6}{\GeV}.
Both of these machines are upgrades from previous versions built on the same sites. 
The first BEPC and BES were originally constructed in 1989, while the upgrade to BESII occurred in 1996.
These two sites were closed in 2004 to prepare for the upgrades to the current systems.


In 2009, BEPCII and BESIII began operation with the goal of utilizing greatly increased luminosity.
For example, instead of the single-bunch electron collisions of BEPC, the new design utilized multiple bunch collisions.
BEPCII also utilizes a dual-storage ring for the electrons and positrons, compared to the single-ring available at BEPC.
The improvements provide BEPCII with a design luminosity of \SI{e33}{\lumunits}, two orders of magnitude larger than the previous installation.
This luminosity is optimized for energies just above the $\DDbar$ threshold, as BESIII conducts many precision measurements and rare decay searches around this region.


\section{BEPCII Accelerator}
\label{sec:BEPCII_accelerator}

The setup for collisions in BEPCII begins with bombarding a fixed target with electrons.
This generates high energy photons which interact with the target material to form $\ee$ pairs.
The positrons from these pairs are then separated magnetically.
Using a linear accelerator, these positrons are then injected into the desired storage ring until they reach the desired beam current.
As electrons do not need to be created in this manner, they are instead directly accelerated and injected into the opposite storage ring.
These injections occur at a rate of \SI{50}{\mAmin} for positrons and \SI{200}{\mAmin} for electrons.


To achieve the necessary high beam currents, many bunches of electrons and positrons are packed into the evacuated rings.
During operation, each ring contains 93 bunches spaced \SI{8}{\ns} (\SI{2.4}{\m}) with a length of \SI{1.5}{\cm}.
These provide a beam current of \SI{0.91}{\A} while operating in collision mode.
At the interaction point, each beam is focused using super-conducting quadrupole magnets to compress the beam size to about \SI{5.7}{\um} vertically, while the horizontal beam size is about \SI{380}{\um}.
For collisions, each beam is also angled towards the center of the storage rings with an angle of \SI{11}{\milli\radian}.
This crossing angle provides better resolution for determining particle momenta in the detector.


For a normal run, collisions continue occurring until the instantaneous luminosity falls below useful levels.
While this is typically depleted due to the collisions between the $\ee$ particles, other unwanted interactions (such as those with beam-gas in the storage rings) also reduce these currents. 
When this happens, BEPCII can replenish the beams using top-off injections.
This allows the collider to continue utilizing the remaining particles within the storage rings without dumping the beams completely.
Recycling these leftover electrons and positrons saves considerable time, and allows for more efficient data taking.


\section{BESIII Detector}
\label{sec:BESIII_detector}

Centered around the interaction point of BEPCII, the BESIII detector records information about the particles produced by the resulting collisions.
Each collision occurs within the beam-pipe of the detector, which is used to minimize multiple-scatterings and secondary interactions.
Its inner and outer radii are \SIlist{31.5;57.0}{\mm}, and is pressurized to \SI{5e-10}{\torr}.
Surrounding the entire apparatus is a uniform, \SI{1}{\tesla} magnetic field provided by a super-conducting solenoid with a mean radius of \SI{1.482}{\m} and a length of \SI{3.53}{\m}.
The field points in the $z$-direction, which is along the direction of the $\alel$ beam.
The $x$-direction points towards the center of the storage rings, while the $y$-direction is vertically upwards.
This magnetic field is used to provide better precision on momenta measurements.
An appropriate field strength curves the tracks of charged particles sufficiently to interact with more of the tracking volume, while minimizing those which curl too much to reach all layers of the detector.

% BESIII Detector Image

The BESIII detector is split into four main layers which analyze different aspects for identifying particles.
Starting from the most interior, these layers are the Multi-Layer Drift Chamber (MDC), the Time-of-Flight System (ToF), the Electromagnetic Calorimeter (EMC), and the Muon Identifier (MU).
Using the information provided by each layer, the particles seen in the detector are given a hypothesis for their most likely candidate.
Only particles stable enough to sufficiently traverse the detector are identifiable.
These include electrons ($e$), muons ($\mu$), pions ($\pi$), kaons ($K$), and protons ($p$).
Particles such as $\DO$ and $\Dp$ must be reconstructed from their decays into these constituents.


\subsection{Multi-Layer Drift Chamber}
\label{ssec:detector_mdc}

The purpose of the Multi-Layer Drift Chamber is to determine the momenta and trajectories of charged particles.
Because of the magnetic field encasing the detector, charged particles will travel in helical trajectories.
The direction of travel is used to determine their charge, while the curvature of the track is used to determine their momenta.

The MDC comprised of many layers of tungsten and sense wires to detect the ionization of particles which pass through its gas-filled volume.
The tungsten wires create a constant electric field which causes ionized electrons to drift towards the sense wires.
This field is tuned to a strength which minimizes secondary ionization.
Conversely, the electric field near each of the sense wires is much larger than the rest of the volume.
This forces an avalanche of secondary ionizations in order to create a current in the sense wires.
The amount of energy deposited by this process is proportional to the original ionization levels.
Tracing the path of energy depositions over time allows for the reconstruction of each charged particle trajectory.


The gas used for ionization is a mixture of $60 \%$ helium (He) and $40 \%$ methane ($\text{C}_3\text{H}_8$).
Helium, being chemically inert, will not interact with the ionized electrons used to measure the position and deposited energy.
Its low atomic number, and thus, long radiation length, also minimizes multiple scatterings which degrade the momentum resolution.
Methane, with extra rotational and vibrational degrees of freedom not accessible to Helium, quenches the ionization energy.
Without this effect, the ionization energy would not be diffused, and would degrade the measurements of deposited energy.


In addition to trajectory, the MDC also measures the rate of energy loss over distance for a particle traveling through a material, 
\beq
-\frac{dE}{dx} = 4 \pi N \frac{z^2 e^4}{m_e \beta^2} \left[ \log \left( \frac{2 m_e \beta^2}{I (1 - \beta^2)} \right) - \beta^2 \right],
\eeq
where $N$ is the electron number density of the material, $z$ is the charge of the particle in terms of $e$, the charge of the electron, $m_e$ is the mass of the electron, $\beta$ is the velocity of the particle, and $I$ is the mean excitation potential for electrons in the material being traversed.
This provides a method of distinguishing particle candidates, as this quantity depends on the velocity of the particle.


There are 43 layers of sense wires within the MDC which cover $93\%$ of the $4\pi$ solid angle in the detector.
These wires create a constant electric field 


\subsection{Time-of-Flight System}
\label{ssec:detector_tof}

The purpose of the Time-of-Flight System is to determine the velocity of charged particles.
This is useful for distinguishing particles with similar momenta, but different masses.
It uses information provided by the MDC to determine the probability for each charged track to match the possible particle hypotheses.
Namely, this includes the measured momentum, the expected time interval based on its trajectory, and the mass for each particle hypothesis.
This provides a separation of $3\sigma$ for $K / \pi$ particles with momenta up to \SI{900}{\MeV}.

% p vs. m^2 Plot

The ToF is comprised of two bands of staggered plastic scintillators attached to photomultiplier tubes (PMTs).
These two bands, located at \SIlist{0.81;0.86}{\m} from the beam-pipe, measure a time difference used to determine the speed of each charged particle.
The resolution is about \SI{100}{\ps}, and is largely limited by the scintillation light rise time, as well as fluctuations associated with the PMTs.
The layer is split into two regions, barrel and endcap, which cover the ranges $|\cos\theta| < 0.82$ and $0.85 < |\cos\theta| < 0.95$, respectively.
The former is dual-layer with each containing 88 scintillators of \SI{5}{\cm} thickness arranged in a trapezoidal cross section, while the latter contains two single layers of 48 fan-shaped scintillators.
Between the two are support structures for the MDC as well as other service lines.


\subsection{Electromagnetic Calorimeter}
\label{ssec:detector_emc}

The purpose of the Electromagnetic Calorimeter is to determine the energy deposited by photons.
Since each of the candidates identified in the detector will be relativistic, they are minimum ionizing particles.
This causes each to deposit a relatively constant value of energy, independent of the measured momenta.
However, electrons, with their extremely small mass, will deposit significant amounts of energy due to Bremsstrahlung radiation.
This provides a clear distinction in the detector between electron and muon tracks above \SI{200}{\MeV}.
Energy measurements from the EMC are also useful for identifying neutral particles which decay only to photons, such as $\piO$.

% E_{EMC} vs. p Plot

The EMC is comprised of tellurium-doped cesium iodide (CsI(Tl)) crystals with square front faces attached to a photodiode.
Each of the 6240 crystals are \SI{5.2}{\cm} long on the square edges and \SI{28}{\cm} (15 radiation lengths) deep.
To prevent photons from aligning with the gaps between each crystal, each one is offset with a tilt of \ang{1.5} in the $\phi$-direction and \ang{1.5} to \ang{3} in the $\theta$-direction.
These crystals provide an energy resolution ($\sigma / E$) of $2.5\%$ at \SI{1}{\GeV} and $4\%$ down to \SI{100}{\MeV}.
This is limited by energy not deposited over the length of the crystal, the areas between crystals, and by non-uniform light production.
Additionally, only measurements of energy above \SI{20}{\MeV} are considered, as below this value is indistinguishable from noise.
The position resolution is $\sigma = \SI{0.6}{\cm} / \sqrt{E \; [\si{\GeV}]}$, and is primarily limited by the crystal segmentation.
The layer has an inner radius of \SI{94}{\cm} and a total weight of approximately 24 tons.
It covers the regions $|\cos\theta| < 0.83$ (barrel) and $0.85 < |\cos\theta| < 0.93$ (endcap), but does not well capture the region between the two.


\subsection{Muon Identifier}
\label{ssec:detector_mu}

The purpose of the Muon Identifier is to determine the likelihood of a charged particle being a muon.
Since electrons are significantly lower mass, they deposit virtually all of their remaining energy in the EMC.
Additionally, since muons do not interact strongly, they will penetrate notably further than will pions, kaons, or protons.
This provides a clear indication of a muon when a particle traverses much of the MU layer.
However, due to the magnetic field, only muons with $p > \SI{0.4}{\GeV}$ will be able to traverse deep enough to be identifiable.


The MU is comprised of resistive plate counters (RPC) which are interspersed between the steel plates of the super-conducting solenoid.
Like the other layers, it is split into a barrel and an endcap region.
The barrel has a total thickness of \SI{41}{\cm} including nine RPC layers.
In the endcap, the first RPC layer is after a \SI{4}{\cm} layer of steel, and thus, has only eight RPC layers.


\section{Triggering Systems}
\label{sec:triggering_systems}

In order to maintain a high efficiency for selecting physics events, many backgrounds must be filtered out.
At BESIII, this is done through a triggering system with two-tiers, level 1 (L1) and level 3 (L3), illustrated in Fig. \ref{fig_triggering}.
The filtered background events are primarily from beam-related sources, such as beam-gas or beam-wall interactions, and occur at a rate of about \SI{13}{\MHz}.
To assist with this process, collimators and masks are used to prevent lost electrons from interacting with the detector.
However, there are also other sources of backgrounds, such as cosmic rays, which occur at a rate of about \SI{1.5}{\kHz}.
The total backgrounds must be suppressed to a rate which does not overwhelm the recording of events by the readout systems.
This rate is roughly \SI{2}{\kHz} at the $\jpsi$ peak, and \SI{600}{\Hz} for the $\psip$ when running near peak luminosity.
For Bhabha events ($\bhabha$), which are used for calibration and luminosity measurements, this rate is \SI{800}{\Hz} within detector acceptance.

% Figure for Triggering Process

The first triggering step (L1) reads out every clock cycle (\SI{24}{\ns}) at a rate of \SI{41.65}{\MHz}.
It uses information from the MDC, ToF, and EMC collectively to reduce the rates of beam-related backgrounds to \SI{1.84}{\kHz} and cosmic rays to about \SI{200}{\Hz}.
However, the L1 has a maximum rate of about \SI{4}{\kHz}.
Because of this, when the buffer holding the subdetector data is around \SI{80}{\%} full, L1 triggers are halted until the buffer drops below \SI{~10}{\%} full.
The efficiency of the L1 process is summarized in Table \ref{tab_L1}.

% Table for L1 efficiencies

From the MDC, the L1 gathers information about each charged track.
The main parameter examined is the number of superlayers a track passed through.
These are defined as `short' tracks if they deposit energy in segments of superlayers 3-5, or `long' tracks for superlayers 3-5 and 10.
To ensure a sufficient momentum to reach the outer superlayers while originating the interaction point, a minimum transverse momentum cut is applied to each track.
This cut is \SI{90}{\MeV} and \SI{120}{\MeV} for short and long tracks, respectively.
In addition to the numbers of short and long tracks for an event, the information about back-to-back tracks is also used.


From the ToF, the L1 gathers information about the number of hits in the barrel and end-cap regions.
It also examines the number of back-to-back hits in each of the two regions.
Here, `back-to-back' is defined as having hits within a range of 9 counters on the opposite side of the detector.


From the EMC, the L1 gathers information about the clustering of energies around a local maximum-energy crystal.
This includes the number of isolated clusters, as well as the information about back-to-back hits in the barrel and end-cap.
Additionally, the balance of energy in the $\phi$-direction (barrel) and in the $z$-direction (endcap) is also used.


The subdetector information gathered during L1 is then passed off to an online computer farm (L3) where the event is assembled.
This step reduces backgrounds from a rate of about \SI{2}{\kHz} to about \SI{1}{\kHz}.
Combined with the signal rate at the $\jpsi$ peak (\SI{2}{\kHz}), this corresponds to a total event rate of \SI{3}{\kHz}, or a tape write speed of \SI{40}{MB\per\s}.

